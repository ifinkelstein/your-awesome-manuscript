% Settings for arara
% https://github.com/islandoftex/arara
%!TEX encoding = UTF-8 Unicode
% arara: xelatex
% arara: biber
% arara: xelatex

\documentclass[11pt]{article}
\usepackage[letterpaper, margin=1in]{geometry}

\usepackage{outlines}
\usepackage{enumitem}

%%%%%%%%%%%%%%%%%%%%%%%%%%%%%%
% Bibliography
\usepackage[style=numeric-comp, backend=biber, sorting=none, sortcites=true]{biblatex}
\addbibresource{bibliography.bib}
%%%%%%%%%%%%%%%%%%%%%%%%%%%%%%

%%%%%%%%%%%%%%%%%%%%%%%%%%%%%%
% Font
% serif fonts are more legible
\usepackage{libertine} % resembles an elegant Times New Roman
% \usepackage[sfdefault]{roboto} %sans-serif font option 
%%%%%%%%%%%%%%%%%%%%%%%%%%%%%%


%%%%%%%%%%%%%%%%%%%%%%%%%%%%%%
% Comments
\usepackage{todonotes}
\newcounter{mycomment}
 % indicate who is making a todo comment using this command
\newcommand{\mycomment}[2][]{
    \refstepcounter{mycomment}
    {\todo[inline, color={red!100!green!33},size=\small]{\textbf{[\uppercase{#1}\themycomment]:}#2}}}
%%%%%%%%%%%%%%%%%%%%%%%%%%%%%%

%%%%%%%%%%%%%%%%%%%%%%%%%%%%%%
% SI units
\usepackage{siunitx}
\DeclareSIUnit\molar{\textsc{M}}
\newcommand{\ul}[1]{\SI{#1}{\micro\liter}}
\newcommand{\ml}[1]{\SI{#1}{\milli\liter}}

\newcommand{\angs}[1]{\SI{#1}{\angstrom}}
\newcommand{\angstrom}{\textup{\AA}}
\newcommand{\um}[1]{\SI{#1}{\micro\meter}}

\newcommand{\degC}[1]{\SI{#1}{\celsius}}

\newcommand{\nanog}[1]{\si{#1}{\nano\gram}} %\ng is taken by an internal command
\newcommand{\ug}[1]{\si{#1}{\micro\gram}} %\ng is taken by an internal command
%%%%%%%%%%%%%%%%%%%%%%%%%%%%%%

%%%%%%%%%%%%%%%%%%%%%%%%%%%%%%
% Line spacing and line numbering
\usepackage{setspace}
\onehalfspacing

%% The lineno packages adds line numbers. Start line numbering with
%% \begin{linenumbers}, end it with \end{linenumbers}. Or switch it on
%% for the whole article with 
%% [modulo] numbers every five lines; see doc for arbitrary numbering
\usepackage[modulo]{lineno}
%%%%%%%%%%%%%%%%%%%%%%%%%%%%%%

\usepackage{booktabs}
\usepackage{amsmath}
\usepackage{nicematrix}

\usepackage{longtable} %format tables across multiple pages
\usepackage{booktabs} %nice looking tables

%%%%%%%%%%%%%%%%%%%%%%%%%%%%%%
% for chemical formulae
\usepackage{chemformula}
%%%%%%%%%%%%%%%%%%%%%%%%%%%%%%


% Keywords command
\newcommand{\keywords}[1]
{
  \small	
  \textbf{\textit{Keywords: }} #1
}

%%%%%%%%%%%%%%%%%%%%%%%%%%%%%%
% Manuscript title and author affiliations
\usepackage{authblk} %author block for adding affiliations easily
\setcounter{Maxaffil}{0} % allow any number of affiliations per author
\renewcommand\Affilfont{\itshape\small} % reduce affiliation font size

\title{Manuscript Title}
\author[1, *]{Jane Doe}
\author[1]{John Doe}
\author[1,2,*]{Ilya J. Finkelstein}
\affil[1]{Department of Molecular Biosciences, University of Texas at Austin, Austin, TX, 78712, USA}
\affil[2]{Center for Systems and Synthetic Biology, University of Texas at Austin, Austin, TX, 78712, USA}
\affil[*]{Correspndence: janedoe@utexas.edu; ilya@finkelsteinlab.org}

\date{}  % remove today's date from the title


%%%%%%%%%%%%%%%%%%%%%%%%%%%%%%
\begin{document}

% comment this out to remove list of todos
\listoftodos 

\maketitle
% \linenumbers % uncomment to add line numbers
\begin{abstract} 
\noindent
Aliquam erat volutpat.  Nunc eleifend leo vitae magna.  In id erat non orci commodo lobortis.  Proin neque massa, cursus ut, gravida ut, lobortis eget, lacus.  Sed diam.  Praesent fermentum tempor tellus.  Nullam tempus.  Mauris ac felis vel velit tristique imperdiet.  Donec at pede.  Etiam vel neque nec dui dignissim bibendum.  Vivamus id enim.  Phasellus neque orci, porta a, aliquet quis, semper a, massa.  Phasellus purus.  Pellentesque tristique imperdiet tortor.  Nam euismod tellus id erat.
\end{abstract}

  %% keywords here
\begin{keywords}
keyword1, keyword2, keyword3, etc.
\end{keywords}

\newpage           
\input{introduction.tex}
\input{results.tex}
\section*{Discussion}

% \mycomment[IJF]{this is an inline comment}

\textcolor{red}{($\leftarrow$ this is an in-line comment )}


%%% Local Variables:
%%% mode: latex
%%% TeX-master: "main"
%%% TeX-engine: xetex
%%% reftex-default-bibliography: ("bibliography.bib")
%%% End:

\input{materials.tex}
\section*{Supplemental Information}

Supplemental information includes zzz figures and zzz tables. 
% Github repo? Nucleotide sequence deposition? Any other databases or resources?

\section*{Declarations}

\paragraph{Author Contributions}

zzz, zzz. and I.J.F. conceived the project. zzz and zzz performed all
experiments and analyzed the data. zzz and zzz prepared the figures. I.J.F.
secured the funding. I.J.F. supervised the project. zzz and zzz and I.J.F. wrote the manuscript with input from all co-authors.

\paragraph{Funding}

This work was supported by NIGMS grants zzz (to I.J.F.), the Welch Foundation grant F-1808 (to I.J.F.), and the College of Natural Sciences Catalyst Award for seed funding. 

\paragraph{Declaration of Interests}
zzz and I.J.F. have filed a patent application relating to zzz. 

%%% Local Variables:
%%% mode: latex
%%% TeX-master: "main"
%%% reftex-default-bibliography: ("bibliography.bib")
%%% End:


\newpage 
          
\printbibliography

\pagebreak

\section*{Figures}
\begin{figure}[h!]
    \centering
    \includegraphics[width=1\linewidth]{figures/fig1.pdf}
    \caption{\textbf{Declarative title} (A) Some description. (B) And this is about panel B.}
    \label{fig:fig1}
  \end{figure}

\begin{figure}[h!]
    \centering
    \includegraphics[width=1\linewidth]{figures/fig2.pdf}
    \caption{\textbf{Declarative title} (A) Some description. (B) And this is about panel B.}
    \label{fig:fig2}
  \end{figure}

%%% Local Variables:
%%% mode: latex
%%% TeX-master: "main"
%%% reftex-default-bibliography: ("bibliography.bib")
%%% End:

\end{document}

% Settings for Emacs
%%% Local Variables:
%%% TeX-engine: xetex
%%% mode: latex
%%% TeX-master: t
%%% End:


