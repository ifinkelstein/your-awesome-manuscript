\documentclass[11pt]{article}

\usepackage[letterpaper, margin=1in]{geometry}

\usepackage{outlines}
\usepackage{enumitem}

%% Fonts and math symbols
% serif fonts are more legible
\usepackage{libertine} % resembles an elegant Times New Roman
% \usepackage[sfdefault]{roboto} %sans-serif font option 
\usepackage{amsmath} % math symbols
\usepackage{nicematrix} %improve matrix and table typesetting
%%%%%%%%%%%%%%%%%%%%%%%%%%%%%%

%% Bibliography
\usepackage[style=nature, backend=biber, sorting=none, sortcites=true]{biblatex}
\renewbibmacro{in:}{} % remove the "In: " text for all entry types
\addbibresource{bibliography.bib}
%%%%%%%%%%%%%%%%%%%%%%%%%%%%%%

%% Figures
% Make figures show up at the top of the page, as opposed to vertically centered
% Note: this is a file-global setting.
% https://texfaq.org/FAQ-vertposfp
\makeatletter
\setlength{\@fptop}{0pt}
\makeatother
\usepackage[labelfont=bf]{caption} %bold figure captions
%%%%%%%%%%%%%%%%%%%%%%%%%%%%%%

%% Tables
\usepackage{longtable} %format tables across multiple pages
\usepackage{booktabs} %nice looking tables
%%%%%%%%%%%%%%%%%%%%%%%%%%%%%%

%% Line spacing and line numbering
\usepackage{setspace}
\onehalfspacing
% The lineno packages adds line numbers. 
% Start line numbering with \begin{linenumbers}, end it with \end{linenumbers}. 
% Or switch it on for the whole article with 
% [modulo] numbers every five lines; see doc for arbitrary numbering
% \usepackage[modulo]{lineno}
%%%%%%%%%%%%%%%%%%%%%%%%%%%%%%

% Comments
% Create comments and to do lists, either inline or in the margin 
\usepackage{todonotes}
\newcounter{comment}
% indicate a todo comment using this custom command. User can supply his initials
\newcommand{\mycomment}[2][]{
    \refstepcounter{comment}
    {\todo[inline, color={red!100!green!33},size=\small]{\textbf{[\uppercase{#1}\themycomment]:}#2}}}
%%%%%%%%%%%%%%%%%%%%%%%%%%%%%%

%% SI units
\usepackage{siunitx}
\DeclareSIUnit\molar{\textsc{M}}
\newcommand{\uL}[1]{\SI{#1}{\micro\liter}} %\ul is taken by underline in 
\newcommand{\mL}[1]{\SI{#1}{\milli\liter}}
\newcommand{\angs}[1]{\SI{#1}{\angstrom}}
\newcommand{\angstrom}{\textup{\AA}}
\newcommand{\um}[1]{\SI{#1}{\micro\meter}}
\newcommand{\degC}[1]{\SI{#1}{\celsius}}
\newcommand{\nanog}[1]{\si{#1}{\nano\gram}} %\ng is taken by an internal command
\newcommand{\ug}[1]{\si{#1}{\micro\gram}} 
%%%%%%%%%%%%%%%%%%%%%%%%%%%%%%

%% write chemical formulae quickly.
% examples (see documentation for much more sophisticated examples):
% \ch{2 H2O}
% \ch{2H2O}
\usepackage{chemformula}
%%%%%%%%%%%%%%%%%%%%%%%%%%%%%%

%% Manuscript title, author affiliations, and keywords
\usepackage{authblk} %author block for adding affiliations easily
\setcounter{Maxaffil}{0} % allow any number of affiliations per author
\renewcommand\Affilfont{\itshape\small} % reduce affiliation font size
% Keywords command
\newcommand{\keywords}[1]
{
  \small	
  \textbf{\textit{Keywords:}} #1
}


%%%%%%%%%%%%%%%%%%%%%%%%%%%%%%

%%% Local Variables:
%%% mode: latex
%%% TeX-master: main
%%% End:
